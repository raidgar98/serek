\documentclass[12pt]{article}

% support for images
\usepackage{graphicx}

% font
\usepackage{tgheros}
\renewcommand*\familydefault{\sfdefault}
\usepackage[T1]{fontenc}


% justify
\usepackage{ragged2e}

% indents
\usepackage{indentfirst}
\setlength{\parindent}{0pt}
\setlength{\parskip}{0pt}
\setlength{\JustifyingParindent}{1.25cm}
% source: https://tex.stackexchange.com/a/121230
\renewenvironment{justify}{%
	\trivlist
	\justifying
	\itemindent\JustifyingParindent
	\item\relax
	}{%
	\endtrivlist
}
\justifying

% fonts, margins, etc...
\usepackage{setspace}
\usepackage{authoraftertitle}
\usepackage{tikz}
\usepackage{anyfontsize}
\usetikzlibrary{positioning,calc}

% language
\usepackage{polski}
\usepackage[utf8]{inputenc}
\usepackage[english,polish]{babel}

% bibliography
\usepackage{biblatex}
\addbibresource{../bibliography.bib}

% configure code fields
\usepackage{listings}
\lstdefinestyle{customcpp}{
  belowcaptionskip=1\baselineskip,
  breaklines=true,
  frame=L,
  xleftmargin=\parindent,
  language=C++,
  showstringspaces=false,
  basicstyle=\fontsize{9pt}{9}\ttfamily,
  keywordstyle=\bfseries\color{green!40!black},
  commentstyle=\itshape\color{purple!40!black},
  stringstyle=\color{orange},
}
\lstset{
	backgroundcolor=\color{white},
	language=C++,
	style=customcpp,
	tabsize=2
}

% set margins
\usepackage[
	a4paper,
	left=2.5cm,
	right=2.5cm,
	top=2.5cm,
	bottom=2.5cm,
	bindingoffset=1cm,
	headheight=1.25cm,
	footheight=1.35cm
]{geometry}

% configure header and footer (with page numbering)
\usepackage[lastpage,user]{zref}
\usepackage{fancyhdr}
\renewcommand{\headrulewidth}{0pt}
\fancyhf{}
\pagestyle{fancy}
\cfoot{\thepage\ z \zpageref{LastPage}}

% document info
\title{Implementacja i badanie wydajności serializacji wykonanej w paradygmacie meta-programowania względem standardowych rozwiązań w języku C++}
\author{inż. Krzysztof MOCHOCKI}
\date{}

% configure enumerator
\renewcommand{\labelenumi}{\arabic{enumi}.}
\renewcommand{\labelenumii}{\arabic{enumi}.\arabic{enumii}.}
\renewcommand{\labelenumiii}{\arabic{enumi}.\arabic{enumii}.\arabic{enumiii}.}
\renewcommand{\labelenumiv}{\arabic{enumi}.\arabic{enumii}.\arabic{enumiii}.\arabic{enumiv}.}

% configure sections
\usepackage{titlesec}
\titleformat{\section}{\normalfont\Large\bfseries}{\thesection.}{1em}{}
\titleformat{\subsection}{\normalfont\large\bfseries}{\thesubsection.}{1em}{}
\titleformat{\subsubsection}{\normalfont\normalsize\bfseries}{\thesubsubsection.}{1em}{}

% function for disabling hyphenation
\newcommand{\disablewordwrap}{%
	\setlength{\hsize}{0.9\hsize}
	% \tolerance=1
	\hyphenpenalty=10000
	\hbadness=10000}

% configuration of table of contents
\usepackage{tocloft}
\setcounter{tocdepth}{1}
\renewcommand{\cftsecleader}{\cftdotfill{\cftdotsep}}


% configuration of figure captions
% source: https://tex.stackexchange.com/a/276796
\usepackage{caption}
\captionsetup[figure]{labelfont={bf}, labelformat={default}, labelsep=period, name={Wykres}}

% configure placing of figures in-text
% source: https://tex.stackexchange.com/a/8633
\usepackage{float}

\begin{document}

	% tittle page
	\begin{titlepage}
		\clearpage
		\topskip0pt
		\centering

		\includegraphics[width=5cm, keepaspectratio=true]{./img/black_and_white_polsl_logo.png}

		{\LARGE\bfseries\ PRACA MAGISTERSKA}

		\vspace*{1cm}

		{\LARGE \MyTitle}

		\Large\bfseries\

		\begin{spacing}{1.125}
			\MyAuthor\

			000000000
			\vspace*{1cm}

			Kierunek: Informatyka Przemysłowa

			Specjalność: Programowanie Komputerów

			\vspace*{1cm}

			PROMOTOR

			dr. inż. Adrian SMAGÓR

			\vspace*{0.5cm}

			KATEDRA Informatyki Przemysłowej

			Wydział Inżynierii Materiałowej

			\vspace*{\vfill}

			KATOWICE 2023
		\end{spacing}

		\thispagestyle{empty}
	\end{titlepage}

	% detail page
	{
		\disablewordwrap
		\large

		\vspace*{\vfill}

		{\bfseries\ Tytuł pracy:} \MyTitle

		\vspace*{\vfill}

		{\bfseries\ Streszczenie:} <tu będzie streszczenie>

		\vspace*{\vfill}

		{\bfseries\ Słowa kluczowe:} serializacja, deserializacja, C++, szablony, wydajność, meta-programowanie

		\vspace*{\vfill}

		{\bfseries\ Thesis title:} Implementation and performance testing of serialization performed in the meta-programming paradigm against standard solutions in the C++ language

		\vspace*{\vfill}

		{\bfseries\ Abstract:} <tu będzie streszczenie po angielskiemu>

		\vspace*{\vfill}

		{\bfseries\ Keywords:} serialization, deserialization, C++, templates, performance, meta-programming

		\vspace*{\vfill}

		\thispagestyle{empty}
		\newpage
	}

	% blank page
	{
		 \vspace*{\vfill}
		\thispagestyle{empty}
		\newpage
	}

	% table of contents
	{
		\tableofcontents
		\thispagestyle{fancy}
		\newpage
	}

	% Introduction
	{
		\section{Wstęp}


		Serializacja w językach wysokiego poziomu, takich jak JavaScript czy Python jest, w przypadku standardowych formatów
		reprezentacji danych, echem przeszłości, ponieważ języki te posiadają natywne wsparcie do reprezentacji struktur danych
		zapewnione przez ichniejsze biblioteki standardowe.\newline

		Niestety w momencie powstawania bieżącej pracy programiści języka C++ wciąż muszą uwzględniać problematykę
		serializacji podczas projektowania aplikacji oraz doboru bibliotek. Standard języka C++ w roku 2023 posiada szeroką
		paletę narzędzi do pisania szablonów klas i struktur, niestety nie posiada żadnych narzędzi do łatwego przeglądania
		struktur danych. Jest to oczywiście podyktowane względami wydajnościowymi oraz chęcią zachowania wstecznej
		kompatybilnością z językiem C. Istnieją rozszerzenia
	}

	\subsection{Pod Wstęp}
	\subsubsection{Pod Pod Wstęp}

	\begin{enumerate}
		\item One
		\item Two
		\item Three
		\begin{enumerate}
			\item Three point one
			\begin{enumerate}
			\item Three point one, point one
				\begin{enumerate}
				\item Three point one, point one, point one 1234567890
				\item Three point one, point one, point two
				\end{enumerate}
			\end{enumerate}
		\end{enumerate}
		\item Four
		\item Five
		\end{enumerate}

	% Bibliography
	{
		\newpage
		\section{Bibliografia}
		\printbibliography[heading=none]
	}
\end{document}
